\documentclass[a4paper,11pt]{article}
\usepackage[pdftex]{graphicx}
\usepackage{color}
\usepackage[top=2cm, bottom=2cm, margin=2cm]{geometry}
\usepackage{longtable}
\usepackage{fancyhdr}
\usepackage{tikz}
\usepackage{mirantis}
\usepackage{booktabs}
\usepackage{tabularx}
\usepackage{color}

\begin{document}

\thispagestyle{empty}
\titleGWP{Deployment System}{Configuration file}{0.1}

\clearpage

\pagestyle{fancy}
\thispagestyle{fancy}
\tableofcontents

\newpage

\section{Revision History}
\begin{tabular}{|l|p{4cm}|p{10cm}|}
\hline
{\bf Date} & {\bf Author} & {\bf Comments} \\ 
\hline
10/06/2013 & Mirantis team & Initial Draft \\ 
\hline
\end{tabular}

\section{Introduction}
The main goal of this document is detailed description of each variable that is contained (or may be contained) in configuration file.

\section{Sections and variables}
\textcolor{red}{Red} marked variables are required. This variables should be in configuration file. \textcolor{blue}{Blue} marked variables are optional. 

\begin{Large}
\begin{itemize}

\item \textbf{[esxvcenter]} - VMWace Vcenter access parameters
\begin{itemize}
\item \large \textcolor{red}{ip} - Ip address of VCenter
\item \textcolor{red}{user} - User name
\item \textcolor{red}{passsword} - Password bla
\end{itemize}

\item \textbf{[esx]} - ESX Hardware Server access parameters
\begin{itemize}
\item \large \textcolor{red}{ip} - Ip address of ESX Server
\item \textcolor{red}{user} - User name
\item \textcolor{red}{password} - Password
\end{itemize}

\item \textbf{[settings]} - Lists of VM's and network's names and common settings
\begin{itemize}
\item \large \textcolor{red}{vms} - List of VM's names separated by commas. This names must be using as a section names
\item \textcolor{red}{networks} - List of network's names separated by commas. This names must be using as a section names
\item \textcolor{blue}{default\_iso} - Path to .iso image, which will be used if 'iso' parameter in 'vm\_name' section is not defined
\end{itemize}

\item \textbf{['network\_name']} - Description about network. For each name in 'settings-networks' must defined section with equal name.
\begin{itemize}
\item \large \textcolor{blue}{vlan} - (int) VLAN number; 0 - no VLAN, 1-4094 - VLAN ID, 4095 - All VLANs. Default vaule - 4095
\item \textcolor{blue}{isolated} - (bool) If this parameter is 'True' - system will create private virtual switch for current network; else network will added as portgroup to shared switch, created specifically for this topology. Default value - 'False'
\item \textcolor{blue}{promiscuous} - (bool) If True, enable Promiscuous mode for current network. Default value - False
\item \textcolor{blue}{ports} - (int) If network is 'isolated', this parameter allow to specify the number of emulated ports. Default value - 120

\end{itemize}



\end{itemize}
\end{Large}

\end{document}
