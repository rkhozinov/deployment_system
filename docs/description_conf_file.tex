\documentclass[a4paper,11pt]{article}

\usepackage[pdftex]{graphicx}
\usepackage{color}
\usepackage[top=2cm, bottom=2cm, margin=2cm]{geometry}
\usepackage{longtable}
\usepackage{fancyhdr}
\usepackage{tikz}
\usepackage{mirantis}
\usepackage{booktabs}
\usepackage{tabularx}
\usepackage{color}
%\usepackage{multirow}
\usepackage{datetime}

\def \frst {2.8cm} % first tab width
\def \scnd {1.2cm} % second tab width
\def \thrd {8cm}
\def \frth {3cm}

\def \tnamefsrt{\textbf{Option}}
\def \tnamescnd{\textbf{Type}}
\def \tnamethrd{\textbf{Description}}
\def \tnamefrth{\textbf{Default value}}

\def \version {0.1}

\let\oldsection\section
\let\oldsubsection\subsection
\def\section{\let\subsectionclearpage\empty\oldsection}
\def\subsection{\subsectionclearpage\let\subsectionclearpage\clearpage\oldsubsection}

\begin{document}

\thispagestyle{empty}
\titleGWP{Deployment System}{Configuration file}{\version}

\clearpage

\pagestyle{fancy}
\thispagestyle{fancy}
\tableofcontents

\newpage

\section{Revision History}
\begin{tabular}{|l|p{4cm}|p{10cm}|}
\hline
{\bf Date} & {\bf Author} & {\bf Comments} \\ 
\hline
{\today} & Mirantis team & Initial Draft \\ 
\hline
\end{tabular}

\section{Introduction}
\paragraph{•}
The main goal of this document is detailed description of each variable that is contained (or may be contained) in configuration file.

\section{Sections and variables}
\paragraph{•}
\textcolor{red}{Red} marked variables are required. This variables should be in configuration file. \textcolor{blue}{Blue} marked variables are optional. \\

%\begin{Large}
\subsection{VMWare vCenter access parameters}
\paragraph{•}
\textbf{[esxvcenter]} - VMWare vCenter access parameters section								\\
\begin{tabular}{p{\frst}|p{\scnd}|p{\thrd}|p{\frth}}
\tnamefsrt 		   		   & \tnamescnd   & \tnamethrd           & \tnamefrth 			\\
\hline
\textcolor{red}{ip}        & string & vCenter IP address 	& 	-                      	\\
\textcolor{red}{user}      & string & User name             & 	-                       	\\
\textcolor{red}{passsword} & string & Password              & 	-                       	\\
\end{tabular}

\subsection{VMWare ESX host server access parameters}
\paragraph{•}
\textbf{[esx]} - VMware ESX host server access parameters section\\
\begin{tabular}{p{\frst}|p{\scnd}|p{\thrd}|p{\frth}}
\tnamefsrt 		   		   & \tnamescnd   & \tnamethrd           & \tnamefrth 			\\
\hline
\textcolor{red}{ip}        & string & ESX server IP address & -                       \\
\textcolor{red}{user}      & string & User name             & -                       \\
\textcolor{red}{passsword} & string & Password              & -                       \\
\end{tabular}

\subsection{Common settings}
\paragraph{•}
\textbf{[settings]} - Common settings section\\

\begin{tabular}{p{\frst}|p{\scnd}|p{\thrd}|p{\frth}}
\tnamefsrt 		   		   & \tnamescnd   & \tnamethrd           & \tnamefrth 			\\
\hline
\textcolor{red}{vms}           & string & List of VM's names separated by commas. This names must be using as a section names            & -                       \\
\textcolor{red}{networks}      & string & List of network's names separated by commas. This names must be using as a section names       & -                       \\
\textcolor{blue}{default\_iso} & string & Path to .iso image, which will be used if 'iso' parameter in VM description section is not defined & Non-defined string      \end{tabular}

\subsection{Network description}
\paragraph{•}
\textbf{['network\_name']} - Network description section. For each name in \textbf{[settings] networks} must defined section with equal name.\\

\begin{tabular}{p{\frst}|p{\scnd}|p{\thrd}|p{\frth}}
\tnamefsrt 		   		   & \tnamescnd   & \tnamethrd           & \tnamefrth 			\\
\hline
\textcolor{blue}{vlan}        & int  & VLAN number; 0 - no VLAN, 1-4094 - VLAN ID, 4095 - All VLANs                                                                                                                               & 4095                    \\
\textcolor{blue}{isolated}    & bool & If this parameter is \textbf{'True'} - system will create private virtual switch for current network, otherwise network will be added to shared switch, created specifically for this topology & False                   \\
\textcolor{blue}{promiscuous} & bool & If \textbf{'True'}, enable Promiscuous mode for current network                                                                                                                                       & False                   \\
\textcolor{blue}{ports}       & int  & If network is \textbf{'isolated'}, this parameter allow to specify the number of emulated ports                                                                                                   & 120                     \\
\end{tabular}

\subsection{Virtual machine description}
\paragraph{•}
\textbf{['vm\_name']} - Virtual Machine description section. For each name in \textbf{[settings] vms} must defined section with equal name.\\

\begin{tabular}{p{\frst}|p{\scnd}|p{\thrd}|p{\frth}}
\tnamefsrt 		   		   & \tnamescnd   & \tnamethrd           & \tnamefrth 			\\
\hline
\textcolor{blue}{description} & string & Annotation for VM, showen in vSphere Client & None\\

\textcolor{blue}{cpu}        & int     & The number of CPUs available to a VM, must be greater than 0        & 1 \\
\textcolor{blue}{memory}     & int     & Size of RAM in MB available to a VM, must be greater than 0         & 512 \\
\textcolor{red}{disk\_space} & int     & Size of \textbf{new} hard drive, which will be created with VM. 
                                         If 0 - new hard drive will \textbf{not} be created                  & - \\
\textcolor{blue}{networks}   & string  & List of networks separated by commas, which must be added to VM. 
                                         Already existing port groups will be checked for existence          & None  \\
\textcolor{blue}{iso}        & string  & Path to \textbf{.iso} image for emulating CD-drive 
                                         (e.g. '/vmfs/volumes/example datastore/example\_iso.iso'). 
                                         If not defined, will be used \textbf{[settings] default\_iso}. 
                                         If '\textbf{False}' string - CD-drive will \textbf{not} be added    & Not defined \\
\textcolor{blue}{hard\_drive} & string & Path to existing \textbf{.vmdk } disk files. If defined and files are exists - hard drive will be added after VM creation. & Not defined \\
\textcolor{blue}{device\_type} & const & A device type based on current VM. Needed for automated pre-configuration (e.g. set up ip or default gw). Values and actions: \textbf{'vyatta'}: Via COM: set up IP, default gw, enable SSH and telnet, log off \textbf{'ubuntu\_without\_password'}: work only if root password not specified. Via \textbf{'vnc'}: setting up IP, default gw, log off. \par \textbf{'other'}: no actions & 'other' \\ 
\textcolor{blue}{vnc\_port} & int & Number of VNC port used for pre-configuration via vnc. Must be unique for each VM. Started from 5900. If 0 - VNC access will \textbf{not} be added. & 0 \\
\textcolor{blue}{config\_type} & const & Technology, used in configuration. If device\_type is not \textbf{'other'}, variable is not used. Values:\par \textbf{'com'}: configuration via serial console\par \textbf{'vnc'}: configuration via VNC & based on \textbf{'device\_type'} and \textbf{'vnc\_port'}: if \textbf{'other'} and \textbf{'vnc\_port'} is defined - '\textbf{vnc'}, else \textbf{'com'} \\
\textcolor{blue}{configuration} & string & List of addition commands, which will be sent to VM after automated pre-configuration. Separated by commas. Note: after pre-configuration VM is logged off. \\
\end{tabular}


\subsection{Pre-configured parameters}
\paragraph{•}
Pre-configured parameters. These parameters are required if \textbf{'device\_type'} is not \textbf{'other'}\\

\begin{tabular}{p{\frst}|p{\scnd}|p{\thrd}|p{\frth}}
\tnamefsrt 		   		   & \tnamescnd   & \tnamethrd           & \tnamefrth 			\\
\hline
\textcolor{red}{user}         & string         & User name in guest OS                      & - \\
\textcolor{red}{password}     & string         & Password for user name                     & - \\
\textcolor{red}{external\_interface} & string  & Interface for configuration (e.g. eth0)    & - \\
\textcolor{red}{ip}           & string         & IPv4 for configuration                     & - \\
\textcolor{red}{mask}         & string         & Netmask, format '/\%d' (e.g. '/24')        & - \\
\textcolor{red}{gw}           & string         & Default gateway - IPv4                     & - \\
\end{tabular}

%\end{Large}

\end{document}
